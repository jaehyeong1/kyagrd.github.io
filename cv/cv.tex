%%%%%%%%%%%%%%%%%%%%%%%%%%%%%%%%%%%%%%%%%%%%%%%%%%%%%%%%%%%%%%%%%%%%%%%%
%%%%%%%%%%%%%%%%%%%%%% Simple LaTeX CV Template %%%%%%%%%%%%%%%%%%%%%%%%
%%%%%%%%%%%%%%%%%%%%%%%%%%%%%%%%%%%%%%%%%%%%%%%%%%%%%%%%%%%%%%%%%%%%%%%%

%%%%%%%%%%%%%%%%%%%%%%%%%%%%%%%%%%%%%%%%%%%%%%%%%%%%%%%%%%%%%%%%%%%%%%%%
%% NOTE: If you find that it says                                     %%
%%                                                                    %%
%%                           1 of ??                                  %%
%%                                                                    %%
%% at the bottom of your first page, this means that the AUX file     %%
%% was not available when you ran LaTeX on this source. Simply RERUN  %%
%% LaTeX to get the ``??'' replaced with the number of the last page  %%
%% of the document. The AUX file will be generated on the first run   %%
%% of LaTeX and used on the second run to fill in all of the          %%
%% references.                                                        %%
%%%%%%%%%%%%%%%%%%%%%%%%%%%%%%%%%%%%%%%%%%%%%%%%%%%%%%%%%%%%%%%%%%%%%%%%

%%%%%%%%%%%%%%%%%%%%%%%%%%%% Document Setup %%%%%%%%%%%%%%%%%%%%%%%%%%%%

% Don't like 10pt? Try 11pt or 12pt
\documentclass[11pt,letterpaper]{article}

% The automated optical recognition software used to digitize resume
% information works best with fonts that do not have serifs. This
% command uses a sans serif font throughout. Uncomment both lines (or at
% least the second) to restore a Roman font (i.e., a font with serifs).
%\usepackage{times}
%\renewcommand{\familydefault}{\sfdefault}

% This is a helpful package that puts math inside length specifications
\usepackage{calc}
\usepackage{comment}
\usepackage{amsmath}
\usepackage[numbers]{natbib}
% \usepackage[sorting=ydnt]{biblatex}
% \usepackage[sorting=none,maxnames=10,bibstyle=authoryear,backend=bibtex,uniquename=true]{biblatex}
% \nocite{*}
% \addbibresource{cv.bib}

% % Simpler bibsection for CV sections
% % (thanks to natbib for inspiration)
% \makeatletter
% \newlength{\bibhang}
% \setlength{\bibhang}{1em} %1em}
% \newlength{\bibsep}
%  {\@listi \global\bibsep\itemsep \global\advance\bibsep by\parsep}
% \newenvironment{bibsection}%
%         {\begin{enumerate}{}{%
% %        {\begin{list}{}{%
%        \setlength{\leftmargin}{\bibhang}%
%        \setlength{\itemindent}{-\leftmargin}%
%        \setlength{\itemsep}{\bibsep}%
%        \setlength{\parsep}{\z@}%
%         \setlength{\partopsep}{0pt}%
%         \setlength{\topsep}{0pt}}}
%         {\end{enumerate}\vspace{-.6\baselineskip}}
% %        {\end{list}\vspace{-.6\baselineskip}}
% \makeatother

% Layout: Puts the section titles on left side of page
\reversemarginpar

%
%         PAPER SIZE, PAGE NUMBER, AND DOCUMENT LAYOUT NOTES:
%
% The next \usepackage line changes the layout for CV style section
% headings as marginal notes. It also sets up the paper size as either
% letter or A4. By default, letter was used. If A4 paper is desired,
% comment out the letterpaper lines and uncomment the a4paper lines.
%
% As you can see, the margin widths and section title widths can be
% easily adjusted.
%
% ALSO: Notice that the includefoot option can be commented OUT in order
% to put the PAGE NUMBER *IN* the bottom margin. This will make the
% effective text area larger.
%
% IF YOU WISH TO REMOVE THE ``of LASTPAGE'' next to each page number,
% see the note about the +LP and -LP lines below. Comment out the +LP
% and uncomment the -LP.
%
% IF YOU WISH TO REMOVE PAGE NUMBERS, be sure that the includefoot line
% is uncommented and ALSO uncomment the \pagestyle{empty} a few lines
% below.
%

%% Use these lines for letter-sized paper
\usepackage[paper=letterpaper,
            %includefoot, % Uncomment to put page number above margin
            marginparwidth=1.2in,     % Length of section titles
            marginparsep=.05in,       % Space between titles and text
            margin=1.5cm,             
	    includefoot,
	    footskip=30pt,
            includemp]{geometry}

%% Use these lines for A4-sized paper
%\usepackage[paper=a4paper,
%            %includefoot, % Uncomment to put page number above margin
%            marginparwidth=30.5mm,    % Length of section titles
%            marginparsep=1.5mm,       % Space between titles and text
%            margin=25mm,              % 25mm margins
%            includemp]{geometry}

%% More layout: Get rid of indenting throughout entire document
\setlength{\parindent}{0in}

\usepackage[shortlabels]{enumitem}

%% Reference the last page in the page number
%
% NOTE: comment the +LP line and uncomment the -LP line to have page
%       numbers without the ``of ##'' last page reference)
%
% NOTE: uncomment the \pagestyle{empty} line to get rid of all page
%       numbers (make sure includefoot is commented out above)
%
\usepackage{fancyhdr,lastpage}
\pagestyle{fancy}
%\pagestyle{empty}      % Uncomment this to get rid of page numbers
\fancyhf{}\renewcommand{\headrulewidth}{0pt}
\fancyfootoffset{\marginparsep+\marginparwidth}
\newlength{\footpageshift}
\setlength{\footpageshift}
          {0.5\textwidth+0.5\marginparsep+0.5\marginparwidth-2in}
\lfoot{\hspace{\footpageshift}%
       \parbox{4in}{\, \hfill %
                    \arabic{page} of \protect\pageref*{LastPage} % +LP
%                    \arabic{page}                               % -LP
                    \hfill \,}}

% Finally, give us PDF bookmarks
\usepackage{color}
\usepackage{hyperref}
\definecolor{darkblue}{rgb}{0.0,0.0,0.3}
\hypersetup{colorlinks,breaklinks,
            linkcolor=darkblue,urlcolor=darkblue,
            anchorcolor=darkblue,citecolor=darkblue}

%%%%%%%%%%%%%%%%%%%%%%%% End Document Setup %%%%%%%%%%%%%%%%%%%%%%%%%%%%


%%%%%%%%%%%%%%%%%%%%%%%%%%% Helper Commands %%%%%%%%%%%%%%%%%%%%%%%%%%%%

% The title (name) with a horizontal rule under it
% (optional argument typesets an object right-justified across from name
%  as well)
%
% Usage: \makeheading{name}
%        OR
%        \makeheading[right_object]{name}
%
% Place at top of document. It should be the first thing.
% If ``right_object'' is provided in the square-braced optional
% argument, it will be right justified on the same line as ``name'' at
% the top of the CV. For example:
%
%       \makeheading[\emph{Curriculum vitae}]{Your Name}
%
% will put an emphasized ``Curriculum vitae'' at the top of the document
% as a title. Likewise, a picture could be included:
%
%   \makeheading[\includegraphics[height=1.5in]{my_picutre}]{Your Name}
%
% the picture will be flush right across from the name.
\newcommand{\makeheading}[2][]%
        {\hspace*{-\marginparsep minus \marginparwidth}%
         \begin{minipage}[t]{\textwidth+\marginparwidth+\marginparsep}%
             {\large \bfseries #2 \hfill #1}\\[-0.15\baselineskip]%
                 \rule{\columnwidth}{1pt}%
         \end{minipage}}

% The section headings
%
% Usage: \section{section name}
\renewcommand{\section}[1]{\pagebreak[3]%
    \hyphenpenalty=10000%
    \vspace{1.3\baselineskip}%
    \phantomsection\addcontentsline{toc}{section}{#1}%
    \noindent\llap{\scshape\smash{\parbox[t]{\marginparwidth}{\raggedright #1}}}%
    \vspace{-\baselineskip}\par}

% An itemize-style list with lots of space between items
\newenvironment{outerlist}[1][\enskip\textbullet]%
        {\begin{itemize}[#1,leftmargin=*]}{\end{itemize}%
         \vspace{-.6\baselineskip}}

% An environment IDENTICAL to outerlist that has better pre-list spacing
% when used as the first thing in a \section
\newenvironment{lonelist}[1][\enskip\textbullet]%
        {\begin{list}{#1}{%
        \setlength{\partopsep}{0pt}%
        \setlength{\topsep}{0pt}}}
        {\end{list}\vspace{-.6\baselineskip}}

% An itemize-style list with little space between items
\newenvironment{innerlist}[1][\enskip\textbullet]%
        {\begin{itemize}[#1,leftmargin=*,parsep=0pt,itemsep=0pt,topsep=0pt,partopsep=0pt]}
        {\end{itemize}}

% An environment IDENTICAL to innerlist that has better pre-list spacing
% when used as the first thing in a \section
\newenvironment{loneinnerlist}[1][\enskip\textbullet]%
        {\begin{itemize}[#1,leftmargin=*,parsep=0pt,itemsep=0pt,topsep=0pt,partopsep=0pt]}
        {\end{itemize}\vspace{-.6\baselineskip}}

% To add some paragraph space between lines.
% This also tells LaTeX to preferably break a page on one of these gaps
% if there is a needed pagebreak nearby.
\newcommand{\blankline}{\quad\pagebreak[3]}
\newcommand{\halfblankline}{\quad\vspace{-0.5\baselineskip}\pagebreak[3]}

% Uses hyperref to link DOI
\newcommand\doilink[1]{\href{http://dx.doi.org/#1}{#1}}
\newcommand\doi[1]{doi:\doilink{#1}}

% For \url{SOME_URL}, links SOME_URL to the url SOME_URL
\providecommand*\url[1]{\href{#1}{#1}}
% Same as above, but pretty-prints SOME_URL in teletype fixed-width font
\renewcommand*\url[1]{\href{#1}{\texttt{#1}}}

% For \email{ADDRESS}, links ADDRESS to the url mailto:ADDRESS
\providecommand*\email[1]{\href{mailto:#1}{#1}}
% Same as above, but pretty-prints ADDRESS in teletype fixed-width font
%\renewcommand*\email[1]{\href{mailto:#1}{\texttt{#1}}}

%\providecommand\BibTeX{{\rm B\kern-.05em{\sc i\kern-.025em b}\kern-.08em
%    T\kern-.1667em\lower.7ex\hbox{E}\kern-.125emX}}
%\providecommand\BibTeX{{\rm B\kern-.05em{\sc i\kern-.025em b}\kern-.08em
%    \TeX}}
\providecommand\BibTeX{{B\kern-.05em{\sc i\kern-.025em b}\kern-.08em
    \TeX}}
\providecommand\Matlab{\textsc{Matlab}}

%%%%%%%%%%%%%%%%%%%%%%%% End Helper Commands %%%%%%%%%%%%%%%%%%%%%%%%%%%

%%%%%%%%%%%%%%%%%%%%%%%%% Begin CV Document %%%%%%%%%%%%%%%%%%%%%%%%%%%%

\begin{document}
\makeheading{Ki Yung \; Ahn ~ {\scriptsize\sf Research Fellow
		in the School of Computer Science \& Engineering
		at Nanyang Technological University, Singapore }  }

\section{Contact Information}

% NOTE: Mind where the & separators and \\ breaks are in the following
%       table.
%
% ALSO: \rcollength is the width of the right column of the table
%       (adjust it to your liking; default is 1.85in).
%
\newlength{\rcollength}\setlength{\rcollength}{5.4cm}%
%
\begin{tabular}[t]{@{}p{\textwidth-\rcollength}p{\rcollength}}
%% Samick Green Mansion 509-205
Cyber Security Lab (Blk. N4-B2C-06)
% \href{http://www.cs.pdx.edu/}{Department of Computer Science}
                            &  \email{kyagrd@gmail.com} \\
%% Godeok-ro 210, Myungil-dong, Gangdong-gu
50 Nanyang Avenue
% \href{http://www.pdx.edu/}{Portland State University} PO Box 751
			    % & +1 503 608 7720 \\
			    % & +82 10 2287 3608 \\
			    & +65 8589 1226 \\
%% Seoul, Replublic of Korea (South) 134-782
% Portland, OR  97207-0751  USA
Singapore 639798
	& \href{http://kyagrd.github.io/}{\url{http://kyagrd.github.io/}}
\end{tabular}

%\section{Objective}

%Insert text here if you want to
%\begin{innerlist}
%\item More information and auxiliary documents can be found at\\\url{http://www.tedpavlic.com/facjobsearch/}
%\end{innerlist}

\section{Research Interests}
Security protocol verification using process calcului,
\newline
Executable relational specifications of polymorphic type systems using logic programming,
\newline
Language design to support both convenient programming and
logically consistent reasoning via the Curry--Howard correspondence,
%% (Mendler-style recursion schemes, term-indexed types),
\newline
Extending the Hindley--Milner (HM) type inference for languages
with Mendler-style recursion schemes and GADTs with true term indices, and
\newline
Interfacing with solvers (e.g., SAT, SMT) in automated testing/verification frameworks.

\section{Education}
\href{http://www.pdx.edu}{\textbf{Portland State University}}, Portland, OR, USA
\vspace*{.5mm}\newline
Ph.D. (Advisor: \href{http://www.cs.umn.edu/~sheard/}{Tim Sheard}),
        \href{http://www.cs.pdx.edu/}{Computer Science},
             Dec 2014
\vspace*{-2mm}
\begin{outerlist}
\item[]
  \href{http://archives.pdx.edu/ds/psu/13198}{\emph{The Nax language}: \emph{unifying functional programming
                                            and logical reasoning in}\newline
              \hfill\emph{a language based on Mendler-style recursion schemes and term-indexed types} }
\end{outerlist}
\vspace{.1in}
\href{http://www.kaist.ac.kr}{\textbf{KAIST}}, Daejeon, Republic of Korea
\vspace*{.5mm}\newline
B.S., \href{http://cs.kaist.ac.kr/}{Computer Science} {\small(major)}
  and \href{http://cs.kaist.ac.kr/}{Mathematics} {\small(sub-major)}, Mar 2002

\section{Research Experience \\ and \\ Academic Visits}
\textbf{Research Fellow} \cite{AhnHorTiu17concur} $[\text{\hyperref[u3]{u3}}]$
	\hfill {Jul 2016 ~-~ current}
\begin{innerlist}
\item[] School of Computer Science \& Engineering, Nanyang Technological University, Singapore ~\newline
	PI: Alwen Tiu (Assistant Professor)
\end{innerlist}
\vspace{2mm}
\textbf{Gratuitous Visit}
	~ {\small(Talk info: \url{http://talks.cam.ac.uk/talk/index/60589})}
	\hfill {Sep 2015}
\begin{innerlist}
\item[] Programming Principles and Tools group, Microsoft Research, Cambridge, UK ~\newline
	Host: Claudio Russo (Senior Research Software Development Engineer)
\end{innerlist}
\vspace{2mm}
\textbf{Academic Visit}
	~ {\small(Talk info: \url{http://slides.com/kyagrd/tiperdundee})}
	\hfill {Aug 2015}
\begin{innerlist}
\item[] Programming Languages, Semantics and Logic group,
	University of Dundee, UK ~\newline
        Host: Ekaterina Komendantskaya (Reader)
\end{innerlist}
\vspace{2mm}
\textbf{Visiting Student} \cite{AhnSheFioPit13}
	{\small(Talk info: \url{http://talks.cam.ac.uk/talk/index/33917})}
	\hfill {Sep\,-\,Dec 2011}
\begin{innerlist}
\item[] Computer Laboratory, University of Cambridge, Cambridge, UK ~\newline
        Hosts: Andrew M. Pitts (Professor), Marcelo Fiore (Professor)
\end{innerlist}
\vspace{2mm}
\textbf{NASA Ames MCT Internship} \cite{Ahn13,Ahn10} \hfill {Jun - Sep 2009}
\begin{innerlist}
\item[] Mission Control Technologies at NASA Ames Research Center, CA, USA ~\newline
        Supervisor: Ewen Denney (Senior Computer Scientist)
\end{innerlist}
\vspace{2mm}
\textbf{Research Assistant (Graduate Student)} \cite{Ahn11,Ahn08,Kimmell12,Vilhelm12} \hfill {Sep 2007 - Sep 2013}

\section{Awards}
\emph{Best Paper Award} \cite{AhnHorTiu17concur} at {CONCUR} 2017
\vspace{2mm} \\
\emph{Bronze Medal} in the ACM Student Research Competition (SRC) at ICFP 2012
\vspace{2mm}

\newpage

\section{Teaching Experience}
\textbf{Full-time Lecturer} \hfill {Spring 2016}\\
Electronics and Information Engineering,
Korea University, Sejong City, Korea  \vspace*{-.4em}
\begin{itemize}
\item \href{http://github.com/kyagrd/eien233ds/}
             {EIEN233(02) Data Structures}
	     {\small(lecture in Korean)} \vspace*{-.5em}
\item \href{http://github.com/kyagrd/eien363ca/}
             {EIEN363(03) Computer Architecture}
             {\small(lecture in Korean)} \vspace*{-.5em}
\item \href{http://github.com/kyagrd/eien215em/}
             {EIEN215(02) Engineering Mathematics I}
             {\small(lecture in English)}
\end{itemize}

\textbf{Teaching Assistant} \hfill {Spring and Summer 2007}
\begin{innerlist}
\item[] \href{http://www.pdx.edu/computer-science/cs-106-computing-fundamentals-ii}
             {CS 106: Computing Fundamentals II}
             {\small(Intro. to programming for non-CS majors)} ~\newline
        Computer Science, Portland State University, Portland, OR, USA ~\newline
        Supervisor: Cynthia A. Brown (Emerita Professor)
\end{innerlist}

% \section{Software Projects}
% TIPER - Type Inference Prototyping Engines from Executable Relational specifcations ~\newline
% \url{https://github.com/kyagrd/tiper}
% \vspace*{2mm}\newline
% Mininax - prototype implementation of the Nax language ~\newline
% \url{https://github.com/kyagrd/mininax}
% \vspace*{2mm}\newline
% Haskell programming interface to the Yices SMT solver (through UNIX pipe) ~\newline
% \url{http://hackage.haskell.org/package/yices}

\section{Industry Experience}
\textbf{Formal Verification Software Engineer (Intern)} \hfill {Sep 2013 - Mar 2014}
\begin{innerlist}
\item[] Refactored parts of the Forte system libraries written in FL (a reflective
        functional language for HW design and theorem proving) and also implemented
        specification search by using term rewriting \newline
        Formal Verification Center of Expertise (DTS/FVCoE),
        Intel, Hillsboro, OR, USA ~\newline
        Supervisors: John W. O'Leary, Roope Kaivola (Principal Engineers)
\end{innerlist}
\vspace{2mm}
\textbf{Quantitative Summer Institute (QSI) Associate (Intern)} \hfill {Jun - Aug 2008}
\begin{innerlist}
\item[] Global Modelling and Analytics Group, Credit Suisse, New York, NY, USA ~\newline
        Supervisor: Howard Mansell (Quantitative Strategist)
\end{innerlist}
\vspace{2mm}
\textbf{Internet Storage Service Server Developer} \hfill {Mar 2002 - May 2005}
\begin{innerlist}
\item[] PopFolder: revenue over 10 million USD, over a million users in 2002 ~\newline
        \href{http://www.gretech.com/eng/}{Gretech}, Seoul, Republic of Korea ~\newline
        Supervisor: Keunho Bae (Director)
        \qquad {\tiny Skills:  C/C++, TCP/IP, UNIX, Berkley DB, PostgreSQL}
\end{innerlist}


\section{Translations}
\href{http://kyagrd.github.io/haskell/}{Korean translation (ISBN 9788972808183)} of\\ \indent \qquad
\href{http://www.cs.nott.ac.uk/~gmh/book.html}{\emph{Programming in Haskell}} (ISBN 9780521692694) by Graham Hutton
\vspace*{-2mm}

\section{Academic Services}
Reviewer (Referee)
\begin{innerlist}
\item[] Typed Lambda Calculus and Applications 2015
\item[] Trends in Functional Programming 2013
\item[] Higher-Order and Symbolic Computation (special issue for PEPM 2012)
\item[] NASA Formal Methods Symposium 2011
\end{innerlist}
\vspace*{2mm}
Program Committee
\begin{innerlist}
\item[] Workshop on Coalgebra, Horn Clause Logic Programming and Types 2016
\end{innerlist}

\section{Talks}
\href{http://slides.com/kyagrd/rowpoly-coalpty16}{A Prolog Specification of Extensible Records using Row Polymorphism} (invited talk)
\begin{innerlist}
\item[] \href{http://slides.com/kyagrd/rowpoly-coalpty16}{Workshop on Coalgebra, Horn Clause Logic Programming and Types (CoALP-Ty '16)},
Edinburgh, UK, 28–-29 November 2016
\end{innerlist}

% \section{Selected Publications}
\section{Publications}
\vspace*{-4mm}
\nocite{*}
\begingroup
\renewcommand{\section}[2]{}%
\bibliographystyle{unsrtnat}
%\bibliographystyle{plainnat}
\bibliography{cv}
\endgroup

\section{Upcoming Papers}
$[\textrm{u1}]$ \label{u1}
Mendler-style recursion schemes for mixed-variant datatypes\\ \hspace*{3.8ex}
Ki Yung Ahn, Tim Sheard, and Marcelo Fiore. \\ \hspace*{3.8ex}
(\href{http://www.irit.fr/TYPES2013/Slides/TYPES13Slides\_Sheard\_et\_al.pdf}
      {slides in TYPES 2013 talk},
 \href{https://www.sharelatex.com/project/53d585f0fc24c66d134e82eb}
      {draft})
\vspace*{1mm} \\
$[\textrm{u2}]$ \label{u2}
An executable specification of typing rules for extensible records based on row polymorphism.\\ $~\,~$ \quad 2017. \qquad % \\ \hspace*{3.8ex} 
\url{https://arxiv.org/abs/1707.07872}
\vspace*{1mm} \\
$[\textrm{u3}]$ \label{u3}
Generating Witness of Non-Bisimilarity for the pi-Calculus. \quad 2017. \\ \hspace*{3.8ex}
Ki Yung Ahn, Ross Horne and Alwen Tiu. \quad \url{https://arxiv.org/abs/1705.10908}


\begin{comment}
\section{Summary of \\my Research Contributions}
%% \hspace*{2ex} My main line of research in recent years has been around
%% my PhD thesis:
%% \begin{quote}
%% We can design a functional language that can be used as a logically consistent
%% theorem prover exploiting Curry--Howard correspondence, while maintaining most
%% of the features familiar to functional programmers who are not necessarily
%% highly exposed to dependently-typed proof assistants.
%% \end{quote}
%% There are two advantages of developing such a language. First is
%% the accessibility from a broader user base because of the little learning curve,
%% compared to being exposed to fully dependent type systems in proof assistants.
%% %% Although dependently typed proof assistants based on the Curry--Howard
%% %% correspondence have have many similarities to functional languages,
%% %% there exists some noticeable differences in their type system when compared to
%% %% the typical type systems of functional languages
%% %% (e.g., predicative stratified universe,
%% %% lack of Hindley--Milner-style type inference,
%% %% strict-positivity restriction on datatype declarations,
%% %% and difficulties in type erasure at compile time).
%% Second is the possibility of collecting low-hanging fruit from
%% the existing theories and tools developed around functional languages.

%% \hspace*{2ex}
The contributions of my reseach can be categorized into three areas:
\begin{description}
\item[~~Executable relational specifications of polymorhpic type systems.]~

\hspace*{1ex}
I (in collaboration with Vezzosi) provided an executable specification
\cite{AhnVez16} of an advanced type system that supports
parametric polymorphism in several dimensions including type polymorphism,
type constructor polymorphism, and kind polymorphism (all in rank-1, for type
inference). The specification is succinct, declarative, and it also servers
as a reference implementation for type inference --- key features of Nax
including pattern matching and Mendler-style iteration could be specified
in less than 100 lines of Prolog. The specification is not only short
but also highly readable because of its structural resemblance to
the typing rules typically used for describing type systems on paper.
In addition, this specification can execute queries of type inference
by exploiting backward-chaining of Prolog as well as checking types.
Recently, I was able to identify key ingredients to specify
extensible records  $[\text{\hyperref[u2]{u2}}]$.
% Supporting other features such as first-class polymorphism, modules, and
% GADTs with term indices are left for future work.

\hspace*{1ex}
I have initiated \href{http://kyagrd.github.com/tiper/}{a project called TIPER}
in order to realize the idea from this line of research (i.e., logic programming
for executable specification of type systems) into a practical tool
that automates type system implementation in compiler construction and
plug-ins for development tools. We already have tools  (e.g. Yacc) for
automatically generating parsers from declarative grammar rule specifications
but in lack of tools for automatically generating type system implementations
supporting polymorphic type inference from typing rule specifications.


\item[~~Mendler-style recursion schemes for mixed-variant datatypes.]~

\hspace*{1ex}
My contribution in Mendler-style recursion schemes for mixed-variant datatypes
provides the basis for safely supporting mixed-variant datatype declarations
in the Nax language and future language designs to be inspired by Nax.
Mainstream dependently typed proof assistants disallow
mixed-variant datatype declarations.

\hspace*{1ex}
I (in collaboration with Sheard) discovered a Mendler-style version
\cite{Ahn11} of the cata-morphism over datatypes
with contravariant recursive occurrences\footnote{ Datatypes with both
	covariant and contravariant recursive occurrences are also known as
	\emph{mixed-variant datatypes}. } (Fegaras and Sheard 2009).
% We named our recursion scheme as the Mendler-style Shared--Fegaras iteration
% (in abbr., \textbf{msfit}). 
Discovering Mendler-style counterparts is more than
mere reformulation of already known recursion schemes.
Unlike conventional recursion schemes, Mendler-style recursion schemes
naturally generalize non-regular datatypes including nested datatypes and
GADTs. In addition, previous studies on Mendler-style recursion schemes
were mostly focused on datatypes with covariant recursive occurrences.

\hspace*{1ex}
Recently, I (in collaboration with Fiore and Sheard)
formulated a Mendler-style version $[\text{\hyperref[u1]{u1}}]$ of
the parametric\footnote{
	As in \emph{parametric} higher-order absract syntax (PHOAS).}
variant of the (non-Mendler-style) Sheard--Fegaras catamorphsim
(Bahr and Hvitved 2012).
However, investigations on the termination properties of
this new recursion scheme are still left for future work.

\item[~~Typed lambda calculi with erasable term indices.]~

\hspace*{1ex}
I (in collaboration with Sheard, Pitts, and Fiore) provided
a baseline understanding of generalized algebraic datatypes (GADTs) with
real term indices and their associated Mendler-style recursion schemes,
by formulating System~$\mathrm{F}_i$ \cite{AhnSheFioPit13}, which extends
System~$\textrm{F}_\omega$ with erasable term indices. System~$\mathrm{F}_i$
is carefully designed to inherit the desirable properties of
System $\mathrm{F}_\omega$ including type preservation and logical consistency.
In my dissertation \cite{AhnPhdThesis14}, I also show that the same extension
can be applied to System~$\mathrm{Fix}_\omega$ (Abel and Matthes 2004)
designed for embedding the Mendler-style primitive recursion.

\hspace*{1ex}
Polymorphic lambda calculi provide us a baseline understanding of
the type systems of functional languages. For example, we know that
extensions to HM such as higher-rank types will not break properties
proven to hold in System~F because higher-rank polymorphism is already
part of System~F. Datatypes and pattern matching can also be understood
within a pure polymorphic lambda calculus via the B\"ohm-Berarducci encoding.
Functional languages with more advanced type systems that support
type constructor polymorphism (a.k.a., higher-kinded polymorphism) could
be understood in terms of System~$\textrm{F}_\omega$.
Due to my contribution of formulating System~$\mathrm{F}_i$,
our baseline understanding now extends over GADTs with real term indicies.
% Further possible extensions to System~$\mathrm{F}_i$ are left for future work.

\end{description}

\vspace*{1.5ex}
In addition to my contributions in the three major areas mentioned
above, I have several other research interests and experiences including
the following items: 
\begin{itemize}
\item During my research internship at NASA Ames, I used QuickCheck
	to generate test cases for the first-order axioms for proving
	automatically generated verification conditions \cite{Ahn10,Ahn13}.
	The main challenge was to cut down the vacuously satisfiable
	tests caused by the variables inside the premise (i.e., left-hand side
	of the implication).
\item I participated in the Trellys research group \cite{Kimmell12,Vilhelm12},
	which was an inter-institutional project funded by the NSF
	in the USA. The goal of the project was to develop
	a dependently typed language that can be used for both
	logical reasoning and programming.
% The Trellys group designed a language
% \cite{Kimmell:2012:ERP:2103776.2103780,DBLP:journals/corr/abs-1202-2923}
% that has two modes of type checking, one for paradox-free proofs
% (as in proof assistants) and the other for programming
% (as in Turing-complete general purpose programing languages),
% and these two modes of typing judgements can carefully refer to each other.

\end{itemize}

\end{comment}

\end{document}




\section{References}
Available upon request.


Tim Sheard $\langle$\email{sheard@cs.pdx.edu}$\rangle$ (Professor)
\begin{innerlist}
\item[]
Department of Computer Science, Portland State University, USA
\end{innerlist}
\halfblankline

% Ewen Denney $\langle$\email{Ewen.W.Denney@nasa.gov}$\rangle$ (Senior Computer Scientist)
% \begin{innerlist}
% \item[]
% Robust Software Engineering Group, NASA Ames Research Center, USA
% \end{innerlist}
% \halfblankline

Marcelo P. Fiore $\langle$\email{Marcelo.Fiore@cl.cam.ac.uk}$\rangle$ (Professor)
\begin{innerlist}
\item[]
Computer Laboratory, University of Cambridge, UK
\end{innerlist}
\halfblankline

Andrew M. Pitts $\langle$\email{Andrew.Pitts@cl.cam.ac.uk}$\rangle$ (Professor)
\begin{innerlist}
\item[] 
Computer Laboratory, University of Cambridge, UK
\end{innerlist}
\halfblankline

Aaron Stump  $\langle$\email{aaron-stump@uiowa.edu}$\rangle$
\begin{innerlist}
\item[] 
Department of Computer Science, University of Iowa, USA
\end{innerlist}
\halfblankline

Stephanie Weirich $\langle$\email{sweirich@cis.upenn.edu}$\rangle$
\begin{innerlist}
\item[]
Department of Computer \& Information Science, University of Pennsylvania, USA
\end{innerlist}


%%%%%%%%%%%%%%%%%%%%%%%%%% End CV Document %%%%%%%%%%%%%%%%%%%%%%%%%%%%%

%----------------------------------------------------------------------%
% The following is copyright and licensing information for
% redistribution of this LaTeX source code; it also includes a liability
% statement. If this source code is not being redistributed to others,
% it may be omitted. It has no effect on the function of the above code.
%----------------------------------------------------------------------%
% Copyright (c) 2007, 2008, 2009, 2010, 2011 by Theodore P. Pavlic
%
% Unless otherwise expressly stated, this work is licensed under the
% Creative Commons Attribution-Noncommercial 3.0 United States License. To
% view a copy of this license, visit
% http://creativecommons.org/licenses/by-nc/3.0/us/ or send a letter to
% Creative Commons, 171 Second Street, Suite 300, San Francisco,
% California, 94105, USA.
%
% THE SOFTWARE IS PROVIDED "AS IS", WITHOUT WARRANTY OF ANY KIND, EXPRESS
% OR IMPLIED, INCLUDING BUT NOT LIMITED TO THE WARRANTIES OF
% MERCHANTABILITY, FITNESS FOR A PARTICULAR PURPOSE AND NONINFRINGEMENT.
% IN NO EVENT SHALL THE AUTHORS OR COPYRIGHT HOLDERS BE LIABLE FOR ANY
% CLAIM, DAMAGES OR OTHER LIABILITY, WHETHER IN AN ACTION OF CONTRACT,
% TORT OR OTHERWISE, ARISING FROM, OUT OF OR IN CONNECTION WITH THE
% SOFTWARE OR THE USE OR OTHER DEALINGS IN THE SOFTWARE.
%----------------------------------------------------------------------%
